\chapter{Sesión 4}

\begin{exercise}
Consulte en el manual las llamadas al sistema para la creación de archivos
especiales en general (mknod) y la específica para archivos FIFO (mkfifo). Pruebe a ejecutar
el siguiente código correspondiente a dos programas que modelan el problema del
productor/consumidor, los cuales utilizan como mecanismo de comunicación un cauce FIFO.
Determine en qué orden y manera se han de ejecutar los dos programas para su correcto
funcionamiento y cómo queda reflejado en el sistema que estamos utilizando un cauce FIFO.
Justifique la respuesta.
\cppscript{../Sesion4/consumidorFIFO}{consumidorFIFO}
\cppscript{../Sesion4/productorFIFO}{productorFIFO}

\emph{El orden da igual, es posible ejecutar primero el productor, quedando así éste
a la espera de que se ejecute el consumidor y realice una lectura sobre el pipe. O puede
ejecutarse primero en consumidor, quedando así a la espera de que se produzcan datos que
consumir.}

\emph{En el sistema queda reflejado el uso del cauce con la existencia del archivo}
\begin{bashcode}
prw-rw-rw-  1 hkr hkr      0 Dec 16 14:04 ComunicacionFIFO|
\end{bashcode}
\end{exercise}

\begin{exercise}
Consulte en el manual en línea la llamada al sistema pipe para la creación de
cauces sin nombre. Pruebe a ejecutar el siguiente programa que utiliza un cauce sin nombre y
describa la función que realiza. Justifique la respuesta.
\begin{cppcode}
/*
tarea6.c
Trabajo con llamadas al sistema del Subsistema de Procesos y Cauces conforme a
POSIX 2.10
*/
#include<sys/types.h>
#include<fcntl.h>
#include<unistd.h>
#include<stdio.h>
#include<stdlib.h>
#include<errno.h>
int main(int argc, char *argv[])
{
int fd[2], numBytes;
pid_t PID;
char mensaje[]= "\nEl primer mensaje transmitido por un cauce!!\n";
char buffer[80];
pipe(fd); // Llamada al sistema para crear un cauce sin nombre
if ( (PID= fork())<0) {
perror("fork");
exit(1);
}
if (PID == 0) {
//Cierre del descriptor de lectura en el proceso hijo
close(fd[0]);
// Enviar el mensaje a través del cauce usando el descriptor de escritura
write(fd[1],mensaje,strlen(mensaje)+1);
exit(0);
}
else { // Estoy en el proceso padre porque PID != 0
//Cerrar el descriptor de escritura en el proceso padre
close(fd[1]);
//Leer datos desde el cauce.
numBytes= read(fd[0],buffer,sizeof(buffer));
printf("\nEl número de bytes recibidos es: %d",numBytes);
printf("\nLa cadena enviada a través del cauce es: %s", buffer);
}
return(0);
}
\end{cppcode}
\emph{Se crea un cauce sin nombre y seguidamente un proceso hijo. En el proceso hijo se cierra el descriptor de lectura almacenado en \textbf{fd[0]}, ya que 
es el hijo que el que va a escribir en el cauce y por tanto debe de cerrar ese lado del cauce. En el proceso padre se realiza lo contrario, se cierra el descriptor de escritura del cauce, ya que éste sólo va a encargarse de leer lo que escriba el hijo.}
\cppscript{../Sesion4/tarea6}{Tarea6}
\end{exercise}

\begin{exercise}
Redirigiendo las entradas y salidas estándares de los procesos a los cauces
podemos escribir un programa en lenguaje C que permita comunicar órdenes existentes sin
necesidad de reprogramarlas, tal como hace el shell (por ejemplo ls | sort). En particular,
ejecute el siguiente programa que ilustra la comunicación entre proceso padre e hijo a través
de un cauce sin nombre redirigiendo la entrada estándar y la salida estándar del padre y el
hijo respectivamente.
\cppscript{../Sesion4/tarea7}{Tarea7}
\end{exercise}

\begin{exercise}
Compare el siguiente programa con el anterior y ejecútelo. Describa la
principal diferencia, si existe, tanto en su código como en el resultado de la ejecución.
\cppscript{../Sesion4/tarea8}{Tarea8}
\emph{La única diferencia es que se usa la llamada al sistema \textbf{dup2}, que realiza simultáneamente las llamadas a \textbf{close} y \textbf{dup}.}
\end{exercise}

\begin{exercise}
Este ejercicio se basa en la idea de utilizar varios procesos para realizar partes de una
computación en paralelo. Para ello, deberá construir un programa que siga el esquema de
computación maestro-esclavo, en el cual existen varios procesos trabajadores (esclavos)
idénticos y un único proceso que reparte trabajo y reúne resultados (maestro). Cada esclavo
es capaz de realizar una computación que le asigne el maestro y enviar a este último los
resultados para que sean mostrados en pantalla por el maestro.

El ejercicio concreto a programar consistirá en el cálculo de los números primos que hay en
un intervalo. Será necesario construir dos programas, maestro y esclavo, que darán lugar a
tres procesos, 1 maestro y 2 esclavos. Además se tendrá que tener en cuenta la siguiente
especificación:

\begin{itemize}
    \item El intervalo de números naturales donde calcular los número primos se pasará como
argumento al programa maestro. El maestro creará dos procesos esclavos y dividirá
el intervalo en dos subintervalos de igual tamaño pasando cada subintervalo como
argumento a cada programa esclavo. Por ejemplo, si al maestro le proporcionamos el
intervalo entre 1000 y 2000, entonces un esclavo debe calcular y devolver los
números primos comprendidos en el subintervalo entre 1000 y 1500, y el otro esclavo
entre 1501 y 2000. El maestro irá recibiendo y mostrando en pantalla (también uno a
uno) los números primos calculados por los esclavos en orden creciente.
    \item El programa esclavo tiene como argumentos el extremo inferior y superior del
intervalo sobre el que buscará números primos. Para identificar un número primo
utiliza el siguiente método concreto: un número n es primo si no es divisible por
ningún k tal que 2 < k < sqrt(n), donde sqrt corresponde a la función de cálculo de la
raíz cuadrada (consulte dicha función en el manual). El esclavo envía al maestro cada
primo encontrado como un dato entero (4 bytes) que escribe en la salida estándar, la
cuál se tiene que encontrar redireccionada a un cauce sin nombre. Los dos cauces sin
nombre necesarios, cada uno para comunicar cada esclavo con el maestro, los creará
el maestro inicialmente. Una vez que un esclavo haya calculado y enviado (uno a uno)
al maestro todos los primos en su correspondiente intervalo terminará.
\end{itemize}
\cppscript{../Sesion4/definitivo}{Ejercicio5}
\end{exercise}

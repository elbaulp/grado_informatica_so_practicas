\section{Sesión 6}

\begin{exercise}
Implementa un programa que admita tres argumentos. El primer argumento será
una orden de Linux; el segundo, uno de los siguientes caracteres “<” o “>”, y el tercero el
nombre de un archivo (que puede existir o no). El programa ejecutará la orden que se
especifica como argumento primero e implementará la redirección especificada por el
segundo argumento hacia el archivo indicado en el tercer argumento. Por ejemplo, si
deseamos redireccionar la salida estándar de sort a un archivo temporal, ejecutaríamos (el
carácter de redirección > lo ponemos entrecomillado para que no lo interprete el shell y se
coja como argumento del programa):
\cppscript{../Sesion6/src/ej1}{Ejercicio 1}
\end{exercise}

\begin{exercise}
Reescribir el programa que implemente un encauzamiento de dos órdenes pero
utilizando fcntl. Este programa admitirá tres argumentos. El primer argumento y el tercero
serán dos órdenes de Linux. El segundo argumento será el carácter “|”. El programa deberá
ahora hacer la redirección de la salida de la orden indicada por el primer argumento hacia el
cauce, y redireccionar la entrada estándar de la segunda orden desde el cauce.
\cppscript{../Sesion6/src/ej2}{Ejercicio 2}
\end{exercise}
